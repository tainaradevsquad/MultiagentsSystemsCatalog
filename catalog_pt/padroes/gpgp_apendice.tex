\section{\textit{Generalized Partial Global Planning}}\label{appendix:gpgp}

A seguir é apresentado um pseudo-código que simula o fluxo entre os agentes simulando as ações descritas na Seção \ref{sec:gpgp}.

\begin{lstlisting}
package gpgp;

public class GPGP {

	public static void main(String[] args) {

	//At time=0 each agent creates its task structure
	Agent agentA = new Agent();
	agentA.addTask(A11, 3); //addTask(taskName, duration)
	agentA.addTask(A12,3);
	
	Agent agentB = new Agent();
	agentB.addTask(B11,2);
	
	Agent agentC = new Agent();
	agentC.addTask(C11, 4);
	agentC.addTask(C12, 1);
		
	//At time=1, each agent makes its own schedule
	agentA.schedule(A11, 1, 3); //schedule(taskName, startTime, endTime)
	agentA.schedule(A12, 4, 6);
	agentB.schedule(B11, 1, 2);
	agentC.schedule(C12, 1, 1);
	
	//At time=2, gpgp plans the tasks considering the global needs
	GeneralizedPartialGlobalPlanner gpgp = new GeneralizedPartialGlobalPlanner();
	winner = gpgp.generatePlanning(1, 2) //generates the winner for the interval from time 1 to 2
	gpgp.notifyAgents(agentA, agentB, agentC); //notifies all agents about the winner of the interval

	//Consider the winner was agentB...
	agentB.executeTask(B11);
	agentA.busyInterval(1, 2); //busyInterval(startTime, endTime)
	agentC.busyInterval(1, 2);
	
	//agentA reschedules, and agentC schedules task C11 
	agentA.schedule(A11, 5, 7);
	agentA.schedule(A12, 2, 4);
	agentC.schedule(C11, 3, 6);
	
	//At time=3
	winner = gpgp.generateaPlanning(3, 4);
	gpgp.notifyAgents(agentA, agentB, agentC);
	
	//Consider the winner was agentA with task A11...
	agentA.executeTask(A11);
	agentB.busyInterval(5, 7);
	agentC.busyInterval(5, 7);
	
	//agentC reschedules
	agentC.schedule(C11, 7, 10);
	
	//At time=4
	winner = gpgp.generatePlanning(7, 10);
	
	//Consider the winner was agentA with task C11...
	agentC.executeTask(C11);
	
	}
}
\end{lstlisting}